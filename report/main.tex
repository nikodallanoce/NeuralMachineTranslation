\documentclass[12pt, letterpaper]{article}  % change to >11 pt if you like, and change article with report
\usepackage[letterpaper, top=3.71cm, bottom=3.20cm, left=2.86cm, right=2.86cm]{geometry}
\usepackage[utf8]{inputenc}
\usepackage{natbib}
\usepackage{graphicx}
\usepackage{color}
\usepackage{subfig}
\usepackage{float}
\usepackage{hyperref}
\usepackage{url}

\title{\vspace{-2cm}\textbf{Human Language Technologies project report}}
\author{\textbf{\small{\textit{Dalla Noce Niko, Ristori Alessandro}}} \\ % put your full name here
        \small{Master Degree in Computer science.}\\ \small{{n.dallanoce@studenti.unipi.it, a.ristori5@studenti.unipi.it}.} \\  % put your Master Degree here
        \small{Human Language Technologies, Academic Year: 2020/2021} \\
        \small{Date: 10/05/2021} \\
       \textbf{\small{\url{https://github.com/nikodallanoce/HLT}}}
}

\renewcommand\refname{} %remove this line to automatically show the bibliography header

\begin{document}

\nocite{*}  % comment this line to list only the articles you really cite
\date{}
\maketitle
\begin{center}
    \includegraphics[width=0.2\textwidth]{images/unipi.png}\\
    \vspace{0.5cm}
\end{center}
\begin{abstract}
The document presents the problem we wish to address, a possible use case with the specifications necessary for the implementation of the project; we describe how to implement our system of sensors using MQTT and how to store their data in a non-relation database like MongoDB. Then, we show how an user can access the stored data through a Telegram bot. Finally, we highlight some features that can improve our project.
\end{abstract}
%\thispagestyle{empty}
%\newpage
%\tableofcontents
%\listoffigures
%\include{Sections/abstract}
\newpage
\tableofcontents
\newpage
\listoffigures
\section{Introduction}
\subsection{Neural Machine Translation}
%\input{Sections/method}
%\input{Sections/experiments}
%\input{Sections/conclusion}
\bibliography{references}

\end{document}

